




\documentclass[french,11pt]{article}
\usepackage[french]{babel}
\usepackage[french]{layout}
\usepackage[T1]{fontenc}
\usepackage[utf8]{inputenc}
\usepackage[a4paper]{geometry}
\usepackage{units}
\usepackage{bera}
\usepackage{graphicx}
\usepackage{textcase}
\usepackage{fancyhdr}
\usepackage{fp}
\usepackage{longtable}
\usepackage{booktabs}
\usepackage{color}
\usepackage{array}
\usepackage{multirow}
\usepackage[table]{xcolor}

\geometry{verbose,tmargin=4em,bmargin=8em,lmargin=6em,rmargin=6em}
\setlength{\parindent}{0pt}
\setlength{\parskip}{1ex plus 0.5ex minus 0.2ex}

\thispagestyle{fancy}
\pagestyle{fancy}
\setlength{\parindent}{0pt}

\renewcommand{\headrulewidth}{0pt}
\definecolor{hydarkblue}{rgb}{0,0.38,0.62}
\definecolor{hylightblue}{rgb}{0.30,0.33,0.37}
\newcolumntype{M}[1]{>{\raggedleft}p{#1}}

\def\SourceName{%
{{source|latex_escape}}
}

\def\EditDate{%
{{object.issuing_date}}
}

\def\TargetName{% 
{{target|latex_escape}}
}

\def\Number{%
 {{object.number|latex_escape}}  Numéro non généré 
}

\def\DocumentName{% 
{{document_name}} 
}
\newcommand{\HyHeader}[1]{
  \centering{#1}
}

\newcommand{\HySourceName}[1]{
{\Large\textbf{#1
}


\newcommand{\HyDocName}[1]{
\color{hydarkblue}\Huge\textbf{\MakeTextUppercase{#1}}
}

\newcommand{\HyInvoiceHeader}[1]{

        \begin{tabular}{| c | c |}
        \hline Numéro & {#1} \\
        \hline Date & {\EditDate}\\
        \hline Réglement & À 30 jours \\
        \hline 
        \end{tabular}
}

\def\TargetAddress{%
{{billing_address.address|latex_escape}} \\
{{billing_address.city|latex_escape}} {{billing_address.po_box|latex_escape}} \\
}

\cfoot{ \footnotesize{ SIRET : {{source.siret}} - NAF : {{source.naf}} -  RCS : {{source.rcs}} - Numéro TVA : {{source.tvaintra}} } }


\begin{document}



\begin{tabular}{@{}p{0.5\linewidth}M{0.5\linewidth
\HySourceName{\SourceName} & \HyDocName{\DocumentName} \tabularnewline
{{source_address.address|latex_escape}} &  \tabularnewline
{{source_address.city|latex_escape}} {{source_address.po_box|latex_escape}} &  \tabularnewline
\end{tabular}

\vspace{1cm}

\begin{tabular}{@{}p{10cm} l }
\multirow{3}{*} {\HyInvoiceHeader{\Number}} & \textbf{\TargetName} \\
 & {{billing_address.address|latex_escape}} \\
 & {{billing_address.city|latex_escape}} {{billing_address.po_box|latex_escape}} \\
\end{tabular} 

\vspace{2cm}


\begin{longtable}
   {|p{0.25\linewidth}|p{0.25\linewidth}|p{0.25\linewidth}|p{0.25\linewidth}|}
   \hline
    \HyHeader{Description} &  \HyHeader{Qté} &  \HyHeader{PU HT} &  \HyHeader{Montant HT}  \endhead
   \hline
   \endfoot
\hline
\multicolumn{3}{c} {} \tabularnewline \cline{3-4}
\multicolumn{2}{c|} {} & \centering{ Total HT } & \centering{ {{object.total_no_vat}} Euros } \tabularnewline \cline{3-4}
\multicolumn{2}{c|} {} & \centering{ TVA } & \centering{ {{object.total_vat|sub:object.total_no_vat}} Euros } \tabularnewline \cline{3-4}
\multicolumn{2}{c|} {} & \centering{ TTC } & \centering{ {{object.total_vat}} Euros } \tabularnewline \cline{3-4}
\endlastfoot


   \hline
     \centering{ {{line.related_item|latex_escape}} }  \centering{ {{line.on_the_fly_item|latex_escape}} }  & \centering{ {{line.quantity}} } &  \centering{ {{line.unit_price}} } & \centering{ {{line.quantity|mult:line.unit_price}} } \tabularnewline



   \hline
    \centering{ {{line.related_item|latex_escape}} }  \centering{ {{line.on_the_fly_item|latex_escape}} }  & \centering{ {{line.quantity}} } &  \centering{ {{line.unit_price}} } & \centering{ {{line.quantity|mult:line.unit_price}} }  \tabularnewline

\end{longtable}

\vspace{3cm}


Vous pouvez régler par chéque ou par virement bancaire sur le compte suivant : \\
\begin{center}

\begin{tabular}{| c | c | c | c | c |  }
\hline
Banque & Guichet & Numéro de Compte & Clé RIB & Domiciliation \\
\hline
 {{object.payment_info.bank_code|latex_escape}} & {{object.payment_info.counter_code|latex_escape}}  & {{object.payment_info.account_number|latex_escape}} & {{object.payment_info.rib_key|latex_escape}}   & {{object.payment_info.banking_domiciliation|latex_escape}}  \\
\hline
\end{tabular}
\end{center}


\begin{center}
En cas de retard de paiement, seront exigibles, conformément à l'article L 441-6 du code de commerce, une indemnité calculée sur la base de trois fois le taux de l'intérêt légal en vigueur ainsi qu'une indemnité forfaitaire pour frais de recouvrement de 40 euros.
\end{center}

\end{document}


